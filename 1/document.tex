\documentclass[a4paper, 12pt]{report}

\makeatletter
\renewcommand{\@seccntformat}[1]{}
\makeatother
\renewcommand{\labelenumii}{\arabic{enumi}.\arabic{enumii}.}

\usepackage{cmap}
\usepackage[T2A]{fontenc}
\usepackage[utf8]{inputenc}
\usepackage[english, russian]{babel}
\usepackage{hyperref}

\usepackage{graphicx}
\graphicspath{{./images/}}

\author{ГАПОУ БАСК}

\title{Windows server. Практическая работа №1}

\date{\today}

\begin{document} %
	
	\maketitle
	\clearpage

	\section{Комплектация задания}

	\begin{enumerate}
		\item Гипервизор VMware
		\item Виртуальная машина с ОС Windows 10
		\item Виртуальная машина с ОС Windows Server 2019
	\end{enumerate}

	\section{Задачи}
	
	\subsubsection{Настройка SRV1}
	
	\begin{enumerate}
		\item Базовая настройка
		\begin{enumerate}
			\item Переименовать компьютер в SRV1;
			\item Установить первый возможный адрес из адресного пространства 10.10.10.0/24;
			\item Обеспечьте работоспособность протокола ICMP (для использования команды ping).
		\end{enumerate}
	\end{enumerate}

	\begin{enumerate}
		\item Active Directory
		\begin{enumerate}
			\item Сделайте сервер контроллером домена bask-rb.ru.
		\end{enumerate}
	\end{enumerate}

	\begin{enumerate}
		\item DHCP
		\begin{enumerate}
			\item настройте протокол DHCP для автоконфигурации клиентов – в качестве диапазона выдаваемых адресов используйте все не занятые серверами адреса в подсети.
		\end{enumerate}
	\end{enumerate}

	\subsubsection{Настройка CLI1}

	\begin{enumerate}
		\item Базовая настройка
		\begin{enumerate}
			\item Переименовать компьютер в CLI1
			\item Настройте динамическую конфигурацию IP
		\end{enumerate}
	\end{enumerate}

	\clearpage
	
	\section{Решение}
	
	\subsection{Настройка локальной сети VMware}
	
	По умолчанию VMware создает одну NAT сеть между виртуальными машинами и сервером. Нам же нужно создать локальный сегмент между CLI1 и SRV1.
	
	Нажмите правой кнопкой мыши на виртуальную машину, выберете в контекстном меню "Параметры..."
	
	\begin{figure}[h]
		\center{\includegraphics[scale=0.6]{1}}
		\caption{123}
		\label{fig:image}
	\end{figure}

	\clearpage

	Откройте вкладку "Сетевой адаптер", нажмите на "Сегменты локальной сети...".

	\begin{figure}[h]
		\center{\includegraphics[scale=0.6]{2}}
		\caption{123}
		\label{fig:image}
	\end{figure}

	\clearpage

	Добавьте новый сегмент "SRV1-CLI1".
	
	\begin{figure}[h]
		\center{\includegraphics[scale=0.6]{3}}
		\caption{123}
		\label{fig:image}
	\end{figure}

	В настройках виртуальных машин укажите сегмент локальной сети "SRV1-CLI1"
	
	\begin{figure}[h]
		\center{\includegraphics[scale=0.6]{4}}
		\caption{123}
		\label{fig:image}
	\end{figure}
	
	\begin{figure}[h]
		\center{\includegraphics[scale=0.6]{5}}
		\caption{123}
		\label{fig:image}
	\end{figure}
	

\end{document}